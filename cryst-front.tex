%%
%% This is file `sample-sigplan.tex',
%% generated with the docstrip utility.
%%
%% The original source files were:
%%
%% samples.dtx  (with options: `sigplan')
%% 
%% IMPORTANT NOTICE:
%% 
%% For the copyright see the source file.
%% 
%% Any modified versions of this file must be renamed
%% with new filenames distinct from sample-sigplan.tex.
%% 
%% For distribution of the original source see the terms
%% for copying and modification in the file samples.dtx.
%% 
%% This generated file may be distributed as long as the
%% original source files, as listed above, are part of the
%% same distribution. (The sources need not necessarily be
%% in the same archive or directory.)
%%
%%
%% Commands for TeXCount
%TC:macro \cite [option:text,text]
%TC:macro \citep [option:text,text]
%TC:macro \citet [option:text,text]
%TC:envir table 0 1
%TC:envir table* 0 1
%TC:envir tabular [ignore] word
%TC:envir displaymath 0 word
%TC:envir math 0 word
%TC:envir comment 0 0
%%
%%
%% The first command in your LaTeX source must be the \documentclass command.
\documentclass[sigplan,preprint,screen,anonymous,review,submission,timestamp,urlbreakonhyphens,%authordraft
                                                                                             ]{acmart}
%%
%% \BibTeX command to typeset BibTeX logo in the docs
\AtBeginDocument{%
  \providecommand\BibTeX{{%
    \normalfont B\kern-0.5em{\scshape i\kern-0.25em b}\kern-0.8em\TeX}}}

%% Rights management information.  This information is sent to you
%% when you complete the rights form.  These commands have SAMPLE
%% values in them; it is your responsibility as an author to replace
%% the commands and values with those provided to you when you
%% complete the rights form.
\setcopyright{acmcopyright}
\copyrightyear{2022}
\acmYear{2022}
\acmDOI{}

%% These commands are for a PROCEEDINGS abstract or paper.
\acmConference[LICS 2022]{LICS 2022: Thirty-Seventh Annual ACM/IEEE Symposium on Logic in Computer Science}{2--5 August, 2022}{Haifa, Israel}
\acmBooktitle{LICS 2022: ACM/IEEE Symposium on Logic in Computer Science,
  August 02--05, 2022, Haifa, Israel}
\acmPrice{}
\acmISBN{}




%------------------------------
% macros
%------------------------------
\input{macros.tex}
%-------------------------------


%-- packages -------------------------------------------
% 
\usepackage{afterpage}
% \usepackage{amsmath,amsthm}
%\usepackage{amssymb}
\usepackage{float}
  %\restylefloat{figure}
% \usepackage[american]{babel}
\usepackage{bussproofs}
\usepackage{calc}
\usepackage{crossreftools} % https://tex.stackexchange.com/questions/405645/referencing-custom-label-with-enumitem
    \makeatletter
    \newcommand{\optionaldesc}[2]{%
      \phantomsection
      #1\protected@edef\@currentlabel{#1}\label{#2}%
    }
    \makeatother
%
% \usepackage{centernot}
% \usepackage{chngcntr}
% \usepackage[dmyyyy]{datetime}
% %\usepackage[inline]{enumitem}
% \usepackage{esvect}
% \usepackage{fancybox}
% \usepackage{hyperref}
\usepackage{placeins}
% \usepackage[switch]{lineno}
% \usepackage{multirow}
\usepackage{mathtools}
\usepackage{moreenum}
%\usepackage[section]{placeins}
\usepackage{soul}   
% \usepackage{stmaryrd}
% \usepackage{url}
% \usepackage{ushort}
% 
% avoid problems created by `\bigtimes' in mathabx
% (solution from https://tex.stackexchange.com/questions/244888/a-problem-caused-by-mathabx-package)
  \let\savell\ll
  \let\ll\relax
  \let\saveemptyset\emptyset
  \let\emptyset\relax
  \let\savebigtimes\bigtimes
  \let\bigtimes\relax
  \usepackage{mathabx}
  \let\bigtimes\savebigtimes
  \let\ll\savell
  \let\emptyset\saveemptyset
\usepackage{graphicx}
% 
%------------------------------------------------------- 


\usepackage{enumitem}
  \let\savedegree\degree
  \let\degree\relax
  \let\savebigtimes\bigtimes
  \let\bigtimes\relax
  \usepackage{mathabx}
  \let\bigtimes\savedegree


%-- tikz -----------------------------------------------
\input{tikz.tex}
\listfiles
% \usepackage{tikz,fp}
%
% \usetikzlibrary{arrows.meta,arrows,calc,automata,angles,quotes,matrix,%backgrounds,calc,chains,fadings,fit,patterns,snakes,
%                 positioning,shapes.geometric,shapes.symbols,shapes.multipart,through,trees,
%                 decorations.markings,decorations.text
%                 }          
% 
% \pgfdeclaredecoration{funbisim}{final}{
%   \state{final}[width=\pgfdecoratedpathlength]{
%     \draw[->]
%       (0,\pgfdecorationsegmentamplitude+0.1mm) -- +(\pgfdecoratedpathlength,0);
%     \draw[-]
%       (0,-\pgfdecorationsegmentamplitude-0.1mm) -- +(\pgfdecoratedpathlength,0);}}
% \tikzset{
%   funbisim/.style={
%     decoration={funbisim, amplitude=0.25ex},
%     decorate,
%     funbisim options/.style={#1}    
%   }}
% 
% \pgfdeclaredecoration{bisim}{final}{
%   \state{final}[width=\pgfdecoratedpathlength]{
%     \draw[<->]
%       (0,\pgfdecorationsegmentamplitude+0.1mm) -- +(\pgfdecoratedpathlength,0);
%     \draw[-]
%       (0,-\pgfdecorationsegmentamplitude-0.1mm) -- +(\pgfdecoratedpathlength,0);}}
% \tikzset{
%   bisim/.style={
%     decoration={bisim, amplitude=0.25ex},
%     decorate,
%     bisim options/.style={#1}    
%   }}   
% 
% \newcommand\encircle[1]{%
%   \tikz[baseline=(X.base)] 
%     \node (X) [draw, shape=circle, inner sep=-1pt] {\strut #1};} 
%     
% \makeatletter
% \def\calcLength(#1,#2)#3{%
% \pgfpointdiff{\pgfpointanchor{#1}{center}}%
%              {\pgfpointanchor{#2}{center}}%
% \pgf@xa=\pgf@x%
% \pgf@ya=\pgf@y%
% \FPeval\@temp@a{\pgfmath@tonumber{\pgf@xa}}%
% \FPeval\@temp@b{\pgfmath@tonumber{\pgf@ya}}%
% \FPeval\@temp@sum{(\@temp@a*\@temp@a+\@temp@b*\@temp@b)}%
% \FProot{\FPMathLen}{\@temp@sum}{2}%
% \FPround\FPMathLen\FPMathLen5\relax
% \global\expandafter\edef\csname #3\endcsname{\FPMathLen}
% }
% \makeatother    
% 
% 
% \tikzset{
%   my dash/.style={dash pattern=on 5pt off 2pt}
%          }% end of tikzset
%-------------------------------------------------------







%%
%% Submission ID.
%% Use this when submitting an article to a sponsored event. You'll
%% receive a unique submission ID from the organizers
%% of the event, and this ID should be used as the parameter to this command.
%%\acmSubmissionID{123-A56-BU3}

%%
%% The majority of ACM publications use numbered citations and
%% references.  The command \citestyle{authoryear} switches to the
%% "author year" style.
%%
%% If you are preparing content for an event
%% sponsored by ACM SIGGRAPH, you must use the "author year" style of
%% citations and references.
%% Uncommenting
%% the next command will enable that style.
%%\citestyle{acmauthoryear}

%%
%% end of the preamble, start of the body of the document source.
\begin{document}

%%
%% The "title" command has an optional parameter,
%% allowing the author to define a "short title" to be used in page headers.
%\title{Cutting Twin-Crystals as Near-Collapsed Process Interpretations of Regular Expressions}
\title{Milner's Proof System for Regular Expressions Modulo Bisimilarity is Complete}
\subtitle{Crystallization: Near-Collapsing Process Graph Interpretations of Regular Expressions}

%%
%% The "author" command and its associated commands are used to define
%% the authors and their affiliations.
%% Of note is the shared affiliation of the first two authors, and the
%% "authornote" and "authornotemark" commands
%% used to denote shared contribution to the research.
\author{Clemens Grabmayer}
  %\authornote{Both authors contributed equally to this research.}
\email{clemens.grabmayer@gssi.it}
\orcid{0000-0002-2414-1073}
% \author{G.K.M. Tobin}
% \authornotemark[1]
%\email{webmaster@marysville-ohio.com}
\affiliation{%
  \institution{Gran Sasso Science Institute}
  \streetaddress{P.O. Box 1212}
  \city{L'Aquila}
  \state{Abruzzo}
  \country{Italy}
  \postcode{67100 AQ}
}

% \author{Lars Th{\o}rv{\"a}ld}
% \affiliation{%
%   \institution{The Th{\o}rv{\"a}ld Group}
%   \streetaddress{1 Th{\o}rv{\"a}ld Circle}
%   \city{Hekla}
%   \country{Iceland}}
% \email{larst@affiliation.org}

% \author{Valerie B\'eranger}
% \affiliation{%
%   \institution{Inria Paris-Rocquencourt}
%   \city{Rocquencourt}
%   \country{France}
% }

% \author{Aparna Patel}
% \affiliation{%
%  \institution{Rajiv Gandhi University}
%  \streetaddress{Rono-Hills}
%  \city{Doimukh}
%  \state{Arunachal Pradesh}
%  \country{India}}

% \author{Huifen Chan}
% \affiliation{%
%   \institution{Tsinghua University}
%   \streetaddress{30 Shuangqing Rd}
%   \city{Haidian Qu}
%   \state{Beijing Shi}
%   \country{China}}

% \author{Charles Palmer}
% \affiliation{%
%   \institution{Palmer Research Laboratories}
%   \streetaddress{8600 Datapoint Drive}
%   \city{San Antonio}
%   \state{Texas}
%   \country{USA}
%   \postcode{78229}}
% \email{cpalmer@prl.com}

% \author{John Smith}
% \affiliation{%
%   \institution{The Th{\o}rv{\"a}ld Group}
%   \streetaddress{1 Th{\o}rv{\"a}ld Circle}
%   \city{Hekla}
%   \country{Iceland}}
% \email{jsmith@affiliation.org}

% \author{Julius P. Kumquat}
% \affiliation{%
%   \institution{The Kumquat Consortium}
%   \city{New York}
%   \country{USA}}
% \email{jpkumquat@consortium.net}

%%
%% By default, the full list of authors will be used in the page
%% headers. Often, this list is too long, and will overlap
%% other information printed in the page headers. This command allows
%% the author to define a more concise list
%% of authors' names for this purpose.
\renewcommand{\shortauthors}{C. Grabmayer}

%%
%% The abstract is a short summary of the work to be presented in the
%% article.
\begin{abstract}
  \input{abstract-short.tex}
  
%   
%   \cite{miln:1984} \cite{grab:fokk:2020:lics}
%   
%   
%   
%   
%   The central reason why the completeness proof for \BBP\ 
%     does not generalize directly to Milner's system is that
%     \onecharts\ with \LEE\ are not preserved under bisimulation collapse.
%     % It turns out charts that can be refined into \onecharts\ with \LEE\ are not preserved under bisimulation collapse. 
%   To circumnavigate this obstacle,
%     we work with LLEE-structure preserving approximations of the bisimulation collapse.
%   We define a crystallization procedure 
%     that minimizes a \LLEEonechart\ under bisimulation \LLEEpreservingly\
%       into a `crystallized' \onechart\ in which every strongly connected component is either collapsed or is of `\twincrystal' form.
%   Then we use the symmetry that \twincrystal\ \sccs\ exhibit to show that crystallized \onecharts\
%     are `\nearcollapsed', and have `complete solutions', that is, solving expressions at bisimilar vertices can be proved equal. 
%   
  
  
\end{abstract}

%%
%% The code below is generated by the tool at http://dl.acm.org/ccs.cfm.
%% Please copy and paste the code instead of the example below.
%%
\begin{CCSXML}
<ccs2012>
<concept>
<concept_id>10003752.10003753.10003761.10003764</concept_id>
<concept_desc>Theory of computation~Process calculi</concept_desc>
<concept_significance>500</concept_significance>
</concept>
<concept>
<concept_id>10003752.10003766.10003776</concept_id>
<concept_desc>Theory of computation~Regular languages</concept_desc>
<concept_significance>100</concept_significance>
</concept>
</ccs2012>
\end{CCSXML}

\ccsdesc[500]{Theory of computation~Process calculi}
\ccsdesc[100]{Theory of computation~Regular languages}
% 
%%
%% Keywords. The author(s) should pick words that accurately describe
%% the work being presented. Separate the keywords with commas.
\keywords{regular expressions, process algebra, bisimilarity, process graphs, complete proof system} 

%% A "teaser" image appears between the author and affiliation
%% information and the body of the document, and typically spans the
%% page.
% \begin{teaserfigure}
%   \includegraphics[width=\textwidth]{acmart/samples/sampleteaser}
%   \caption{Seattle Mariners at Spring Training, 2010.}
%   \Description{Enjoying the baseball game from the third-base
%   seats. Ichiro Suzuki preparing to bat.}
%   \label{fig:teaser}
% \end{teaserfigure}

%%
%% This command processes the author and affiliation and title
%% information and builds the first part of the formatted document.
\maketitle


\raggedbottom
%\flushbottom

\end{document} 


%--------------------
% article
%--------------------
\input{cryst-art.tex}
%--------------------


%-------------
% bibliography
%-------------
\clearpage
%%
%% The next two lines define the bibliography style to be used, and
%% the bibliography file.
\bibliographystyle{ACM-Reference-Format}
\bibliography{cryst}



%% If your work has an appendix, this is the place to put it.
% \appendix

\newpage\onecolumn%\mbox{}
%% Appendix
%=================== 
\appendix%
\section{Appendix} %{: supplements, more proof details, and omitted proofs}%
%\section*{S \mbox{} Supplements (Appendix)}%
  \label{appendix}%
%===================
%\addtocounter{section}{19}  
% \footnotetext{\hspace*{0.5pt}%
%   {\nf\bf Guidelines appendix.}
%       `If necessary, detailed proofs of technical results may be included in a clearly-labeled appendix, 
%       to be consulted at the discretion of program committee members.'}    




\newpage

%----------
\subsubsection{Definitions of \LEEbf\ and \LLEEbf\ for \onecharts} \mbox{}
  \label{def::LLEEonecharts::app}
%----------
 
\smallskip\noindent
%
In this subsection we recall principal definitions and statements from \cite{grab:fokk:2020:lics,grab:2021:TERMGRAPH2020-postproceedings}.
We keep formalities to a minimum as necessary for our purpose (in particular for `\LLEEwitnesses').  
  
\smallskip 
  
% \begin{defi}[loop 1-chart]\label{def:loop:chart}\nf
  A \onechart~$\aoneloop = \tuple{\verts,\actions,\sone,\start,\transs,\termexts}$ is called a \emph{loop \onechart} if
  it satisfies three conditions:
  \begin{enumerate}[label={{\rm (L\arabic*)}},leftmargin=*,align=left,itemsep=0.5ex]
    \item{}\label{loop:1}
      There is an infinite path from the start vertex $\start$.
    \item{}\label{loop:2}  
      Every infinite path from $\start$ returns to $\start$ after a positive number of transitions.
      % (and so visits $\start$ infinitely often).
    \item{}\label{loop:3}
      Immediate termination is only permitted at the start vertex, that is, $\termexts\subseteq\setexp{\start}$.
  \end{enumerate}
  We call the transitions from $\start$ \emph{\loopentry\ transitions},
  and all other transitions \emph{\loopbody\ transitions}.
  %
  A \emph{loop \subonechart\ of} a \onechart~$\aonechart$
    is a loop \onechart~$\aoneloop$
    that is a \subonechart\ of $\aonechart$ 
      with some vertex $\avert\in\verts$ of $\aonechart$ as start vertex,
    such that $\aoneloop$ is constructed, for a nonempty set $\asettranss$ of transitions of $\aonechart$ from $\avert$,
    by all paths that start with a transition in $\asettranss$ and continue onward until $\avert$ is reached again
  (so the transitions in $\asettranss$ are the \loopentrytransitions~of~$\aoneloop$).
% Within the chart interpretation of a `\onefree'~\cite{grab:fokk:2020a} star expression~$\astexp$,
% loop-\subonecharts\ correspond to the behavior of $\stexpit{\bstexp}$ within innermost subterms~$\stexpit{\bstexp}$~in~$\astexp$.  
%  
% \end{defi}
%
%   A \emph{loop subgraph} of a process graph is generated from a set $E$ of entry transitions from a vertex $\avert$ 
% by all paths from $\avert$ that start along a transition in $E$
% and continue until $\avert$ is reached again first,
% given that three properties hold for the so-constructed subgraph:
%
% Both of the not expressible charts $\chartnei{1}$ and $\chartnei{2}$ in Ex.~\ref{ex:chart:interpretation} are not loop charts:
% $\chartnei{1}$ violates \ref{loop:3}, and $\chartnei{2}$ violates \ref{loop:2}.
% Moreover, none of these charts contains a loop subchart.
% The chart $\chartof{\cstexpi{0}}$ in Ex.~\ref{ex:chart:interpretation} is not
% a loop chart either, as it violates \ref{loop:2}. But we will see that $\chartof{\cstexpi{0}}$ has loop subcharts. 
%
%
% The property \LEE\ is defined by a dynamic elimination procedure (see below for an example) 
%   that tests the structure of a \onechart:
%   by peeling off `loop sub-{}\onecharts' as long as possible
%   a finite \onechart\ without infinite paths (without cycles) must be reachable. 
% Within the chart interpretation of a `\onefree'~\cite{grab:fokk:2020a} star expression~$\astexp$,
% loop-\subonecharts\ correspond to the behavior of $\stexpit{\bstexp}$ within innermost subterms~$\stexpit{\bstexp}$~in~$\astexp$.

The result of \emph{eliminating a loop \subonechart\ $\aoneloop$ from a \onechart\ $\aonechart$}
  arises by removing all \loopentrytransitions\ of $\aoneloop$ from $\aonechart$, 
  and then also removing all vertices and transitions that become unreachable. 
  %
  We say that a \onechart\ $\aonechart$ has the \emph{loop existence and elimination property} (\LEE)
  if the procedure, started on~$\aonechart$, of repeated eliminations of loop \subonecharts\
  results in a \onechart\ without an infinite path.
  If, in a successful elimination process from a \onechart~$\aonechart$,
  \loopentrytransitions\ are never removed from the body of a previously eliminated loop \subonechart,
  then we say that $\aonechart$ satisfies \emph{layered \LEE} (\LLEE),
  and is a \emph{\LLEEonechart}. 
  %  
While the property \LLEE\ leads to a formally easier concept of `witness', it is equivalent to %the property 
                                                                                               \LEE. 
(For an example of a \LEEwitness\ that is not layered, see further below on page~\pageref{non-ex-LLEEw}.)                                                                                    
% This restriction of the elimination procedure does not  the property \LEE,
% but leads to a formally easier concept. % that is easier to reason about.

\begin{center}
  \input{figs/ex-LEE-app.tex}%
\end{center}

The picture above shows a successful run of the loop elimination procedure. % for the \onechart~$\aonechart$.
In brown we highlight start vertices by \picarrowstart, and immediate termination with a boldface ring.
The \loopentry\ transitions of loop \subonecharts\ that are eliminated in the next step are marked in bold. 
We have neglected action labels here, except for indicating \onetransitions\ by~dotted~arrows.
% \begin{center}
%   \input{figs/ex-LEE.tex}
% \end{center}\vspace*{-1ex}%
Since the graph $\aonechart'''$ that is reached after three loop-subgraph elimination steps from the \onechart\ $\aonechart$ does not have 
an infinite path, and no \loopentry\ transitions have been removed from a previously eliminated loop \subonechart,
we conclude that $\aonechart$ satisfies \LEE\ and \LLEE. 
 
\begin{center}
  \input{figs/ex-LLEEw-1-2-3-app.tex}
\end{center}
%
A \emph{\LLEEwitness\ $\aonecharthat$ of} a \onechart~$\aonechart$
is the recording of a %\LLEE-guaranteeing, 
                          successful run of the loop elimination procedure
by attaching to a transition $\atrans$ of $\aonechart$ the marking label $n$ for $n\in\natplus$ 
 (in pictures indicated as $\looplab{n}$, in steps as $\sredi{\looplab{n}}$) 
 forming a \emph{\loopentry\ transition}
if $\atrans$ is eliminated in the $n$\nb-th step,
and by attaching marking label $0$ to all other transitions of $\aonechart$
 (in pictures neglected, in steps indicated as $\sredi{\bodylab}$)
 forming a \emph{body transition}. 
Formally, \LLEEwitnesses\ arise as \emph{\entrybodylabeling{s}} from \onecharts,
 and are charts in which the transition labels are pairs of action labels over $\actions$,
 and marking labels in $\nat$.
We say that a \LLEEwitness~$\aonecharthat$ \emph{is guarded}
 if all \loopentrytransitions\ are proper, which means that they have a proper-action~transition~label.
  
The \entrybodylabeling\ $\aonecharthati{1}$ above of the \onechart~$\aonechart$ %from Ex.~\ref{ex:LEE}
is a \LLEEwitness\ that arises from the run of the loop elimination procedure earlier above. % in Ex.~\ref{ex:LEE}.
%
The \entrybodylabelings\ $\aonecharthati{2}$ and $\aonecharthati{3}$ of $\aonechart$ record two other successful runs of the loop elimination procedure
of length 4 and 2, respectively, where for $\aonecharthati{3}$ we have permitted to eliminate two loop subcharts at different vertices
together in the first step. 
The \onechart~$\aonechart$ only has layered \LEEwitnesses.
But that is not the case for the \onechart~$\conechart$ below:

\begin{center}\vspace{-0.75ex}\label{non-ex-LLEEw}%
  \scalebox{0.95}{\input{figs/non-ex-LLEEw.tex}}\vspace*{-1.5ex}
\end{center}
%
The \entrybodylabeling~$\conecharthati{1}$ of $\conechart$ as above is a \LEEwitness\ that is not layered:
  in the third loop \subonechart\ elimination step that is recorded in $\conecharthati{1}$
  the \loopentrytransition\ from $\bverti{1}$ to $\bverti{2}$ is removed
  that is in the body of the loop \subonechart\ at $\avert$ with \loopentrytransition\ from $\avert$ to $\bverti{1}$,
    which (by that time) has already been removed in the first \loopelimination\ step as recorded in $\conecharthati{1}$.
But the \entrybodylabeling~$\conecharthati{2}$ of $\conechart$ above is a layered \LEEwitness.
It can be shown that \LEEwitnesses\ that are not layered can always be transformed into \LLEEwitnesses\ of the same underlying \onechart.    
Indeed, the step from $\conecharthati{1}$ to $\conecharthati{2}$ in the example above,
  which transfers the \loopentrytransition\ marking label $\loopsteplab{3}$
  from the transition from $\bverti{1}$ to $\bverti{2}$ over to the transition from $\avert$ to $\cvert$,
hints at the proof of this general statement.
We do, however, not need this result, because
  we will be able to use the guaranteed existence of \LLEEwitnesses\ (see Thm.~\ref{thm:onechart-int:LLEEw})
  for the \onechart\ interpretation below (see Def.~\ref{def:onechartof}).
    
    
    
    
\end{document}    
\endinput
%%
%% End of file `sample-sigplan.tex'.


